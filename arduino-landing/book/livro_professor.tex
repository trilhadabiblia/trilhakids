% livro_professor.tex
% Compilar com: xelatex livro_professor.tex (2x)
\documentclass[12pt,a4paper]{book}
\usepackage{fontspec}
\usepackage[brazil]{babel}
\usepackage{microtype}
\usepackage{graphicx}
\usepackage{hyperref}
\usepackage{longtable}
\usepackage{tabularx}
\usepackage{listings}
\usepackage{enumitem}
\usepackage{caption}
\usepackage{float}
\usepackage{geometry}
\usepackage{xcolor}
\geometry{margin=2.5cm}

% Código Arduino settings
\lstset{
  language=C,
  basicstyle=\ttfamily\small,
  breaklines=true,
  frame=single,
  numbers=left,
  numberstyle=\tiny,
  keywordstyle=\color{blue},
  commentstyle=\color{gray},
  morekeywords={pinMode, digitalWrite, analogRead, analogWrite, delay, tone, noTone}
}

% Metadata
\title{Arduino Kids — Guia do Professor}
\author{Projeto Arduino Kids \\ Desenvolvido por Claudecir Miranda}
\date{\today}

\begin{document}
\maketitle
\thispagestyle{empty}
\clearpage

\tableofcontents
\clearpage

\chapter*{Apresentação}
\addcontentsline{toc}{chapter}{Apresentação}
Breve apresentação do programa, público-alvo, objetivos gerais, alinhamento BNCC e como utilizar este guia.

\chapter{Projeto Pedagógico}
\section{Objetivos Gerais}
\begin{itemize}
  \item Desenvolver pensamento lógico e criativo.
  \item Introduzir conceitos básicos de eletrônica e programação.
  \item Integrar temas transversais (meio ambiente, cidadania).
\end{itemize}

\section{Metodologia}
Descrição da abordagem — aprendizagem por projetos, divisão em estações, papéis (montador, programador, narrador), avaliação formativa.

\section{Avaliação}
Instrumentos sugeridos: observação, checklist, portfólio do aluno, rubricas de criatividade e resolução de problemas.

\chapter{Como usar os projetos}
Instruções de logística: tempo por projeto, número de kits, segurança, lista de materiais por turma.

\section{Lista geral de materiais por kit}
\begin{longtable}{p{8cm} p{5cm}}
\textbf{Componente} & \textbf{Quantidade por kit} \\
\hline
Arduino Uno (ou compatible) & 1 \\
Protoboard pequena & 1 \\
Jumpers macho-fêmea & 20 \\
LEDs (vermelho, amarelo, verde) & 3 \\
Resistores 220\(\Omega\) & 3 \\
Buzzer piezo & 1 \\
Sensor de umidade & 1 \\
Sensor ultrassônico HC-SR04 & 1 \\
OLED 0.96" (opcional) & 1 \\
% ... adicione o restante
\end{longtable}

\chapter{Projetos - Guia do Professor}
% Repetir o bloco abaixo para cada um dos 10 projetos
\section{Projeto 1 — Semáforo Infantil (6--8 anos)}
\subsection{Objetivos}
\begin{itemize}
  \item Ensinar sequência e noções de trânsito.
  \item Trabalhar atenção e tempo.
\end{itemize}

\subsection{Tempo estimado}
45 minutos (30 min montagem + 15 min atividade prática)

\subsection{Materiais por grupo}
\begin{itemize}
  \item 1 Arduino Uno
  \item Protoboard
  \item LEDs: vermelho, amarelo, verde
  \item Resistores 220\(\Omega\) x3
  \item Buzzer
  \item Jumpers
\end{itemize}

\subsection{Preparação do professor}
Dicas de preparo: pré-montar partes, verificar códigos, imprimir fichas.

\subsection{Passo a passo (sugestão para condução)}
\begin{enumerate}
  \item Apresentar a história do personagem (Luzinha).
  \item Demonstrar montagem do circuito (mostrar no projetor).
  \item Dividir os alunos em grupos e distribuir materiais.
  \item Orientar montagem e testar com o código.
  \item Propor extensão: adaptar tempo ou botão de pedestre.
\end{enumerate}

\subsection{Código exemplo}
\begin{lstlisting}
// Semáforo simples
int ledV = 10;
int ledA = 9;
int ledG = 8;
int buzzer = 7;

void setup() {
  pinMode(ledV, OUTPUT);
  pinMode(ledA, OUTPUT);
  pinMode(ledG, OUTPUT);
  pinMode(buzzer, OUTPUT);
}

void loop() {
  digitalWrite(ledV, HIGH); tone(buzzer, 500); delay(1000); noTone(buzzer); digitalWrite(ledV, LOW);
  digitalWrite(ledA, HIGH); delay(700); digitalWrite(ledA, LOW);
  digitalWrite(ledG, HIGH); delay(1000); digitalWrite(ledG, LOW);
}
\end{lstlisting}

\subsection{Sugeridas adaptações e desafios}
\begin{itemize}
  \item Inserir botão de pedestre (material adicional).
  \item Fazer competição de quem monta mais rápido com qualidade.
  \item Trabalho interdisciplinar: mapa de rua em papelão.
\end{itemize}

\subsection{Competências BNCC}
Relacionar explicitamente os itens da BNCC atendidos (ex.: EF15CI01, etc.) — preencher com os códigos apropriados.

\subsection{Avaliação}
Rubrica com critérios: montagem correta, explicação do circuito, trabalho em equipe, criatividade.

\vspace{6pt}
% Fim do projeto 1. Repetir para projetos 2..10

\chapter{Apêndices}
\section{Códigos completos}
Coleção de códigos de todos os projetos (copiar e colar os \texttt{lstlisting}).

\section{Esquemáticos}
Instruções para inserir imagens: coloque os PDFs/PNGs em \texttt{images/} e inclua com \verb|\includegraphics[width=0.8\textwidth]{images/semaforo.png}|.

\section{Modelos de Ficha de Avaliação}
Sugestões de rubricas em tabela.

\chapter*{Referências}
\addcontentsline{toc}{chapter}{Referências}
Listar referências pedagógicas, links e agradecimentos.

\end{document}
