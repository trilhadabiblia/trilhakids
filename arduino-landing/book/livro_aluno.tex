% livro_aluno.tex
% Compilar com: xelatex livro_aluno.tex (2x)
\documentclass[11pt,a4paper]{article}
\usepackage{fontspec}
\usepackage[brazil]{babel}
\usepackage{graphicx}
\usepackage{hyperref}
\usepackage{multicol}
\usepackage{enumitem}
\usepackage{geometry}
\geometry{margin=2.2cm}

\title{Arduino Kids — Caderno do Aluno}
\author{Projeto Arduino Kids}
\date{}

\begin{document}
\maketitle
\thispagestyle{empty}
\clearpage

\tableofcontents
\clearpage

\section*{Como usar este caderno}
Explicação curta e simples: siga as etapas, registre suas experiências, desenhe e cole fotos.

\section{Missão 1 — Semáforo Infantil (6--8 anos)}
\textbf{Personagem:} Luzinha \\
\textbf{Tempo:} 45 minutos \\
\textbf{Objetivo:} Aprender sobre cores e sequência.

\subsection*{1) Materiais}
Lista simples (LEDs, resistores, Arduino, buzzer).

\subsection*{2) Montagem}
Espaço para colar o esquema (deixe um retângulo):
\begin{center}
\fbox{\parbox[c][6cm][c]{0.8\textwidth}{\centering Cole aqui a foto do seu circuito}}
\end{center}

\subsection*{3) Código}
\begin{verbatim}
// Semáforo simples (versão curta)
void setup(){ pinMode(10, OUTPUT); pinMode(9, OUTPUT); pinMode(8, OUTPUT);}
void loop(){ digitalWrite(10, HIGH); delay(1000); digitalWrite(10, LOW);
 digitalWrite(9, HIGH); delay(700); digitalWrite(9, LOW);
 digitalWrite(8, HIGH); delay(1000); digitalWrite(8, LOW); }
\end{verbatim}

\subsection*{4) Perguntas (responda em poucas palavras)}
\begin{enumerate}[label=\alph*)]
  \item O que aconteceu quando você ligou o semáforo?
  \item Qual cor significa que podemos passar?
  \item O que você aprendeu de novo?
\end{enumerate}

\subsection*{5) Desafio Criativo}
Desenhe um carro que segue o semáforo ou invente uma história curta sobre Luzinha.

\clearpage

% Repetir blocos semelhantes para as demais missões/projetos (2..10)
\section{Missão 2 — Planta que Fala (6--8 anos)}
% ... conteúdo similar (materiais, montagem, espaço para foto, perguntas)

\section*{Glossário}
Definições simples: LED, resistor, sensor, protoboard.

\section*{Certificado}
Espaço para carimbo / assinatura quando completar todas as missões.

\end{document}
